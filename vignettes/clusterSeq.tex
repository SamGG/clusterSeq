%\VignetteIndexEntry{Advanced baySeq analyses}
%\VignettePackage{baySeq}
%\VignetteKeywords{baySeq, generic, advanced}

\documentclass[a4paper]{article}

\title{Clustering high-throughput sequencing data based on patterns of co-expression}
\author{Thomas J. Hardcastle}

\RequirePackage{/home/bioinf/tjh48/R/x86_64-pc-linux-gnu-library/3.3/BiocStyle/resources/tex/Bioconductor}

\AtBeginDocument{\bibliographystyle{/home/bioinf/tjh48/R/x86_64-pc-linux-gnu-library/3.3/BiocStyle/resources/tex/unsrturl}}
\usepackage{Sweave}
\begin{document}

\maketitle

\section{Introduction}

This vignette outlines two possibilities for clustering gene expression based on patterns of co-expression between samples, optionally accounting for the replicate structure of the data.

The package can be installed as
\begin{Schunk}
\begin{Sinput}
> source("http://www.bioconductor.org/biocLite.R")
> biocLite("clusterSeq")
\end{Sinput}
\end{Schunk}

\begin{Schunk}
\begin{Sinput}
> library(clusterSeq)
\end{Sinput}
\end{Schunk}

% data are filtered for minimum expression
%#

We demonstrate these analyses on a set of time series data in female rat thymus tissues taken from the Rat Bodymap project \cite{Yu2014}. By identifying clusters of genes that demonstrate similar patterns of expression over time we can identify patterns of time dependence within the data.

We first load in the processed data of observed read counts at each gene for each sample. 
\begin{Schunk}
\begin{Sinput}
> data(ratThymus, package = "clusterSeq")
> head(ratThymus)
\end{Sinput}
\end{Schunk}

We define the replicate structure of the data in a vector whose members correspond to the columns of the data matrix.
\begin{Schunk}
\begin{Sinput}
> replicates <- c("2week","2week","2week","2week",
+                 "6week","6week","6week","6week",
+                 "21week","21week","21week","21week",
+                 "104week","104week","104week","104week")
\end{Sinput}
\end{Schunk}

Library scaling factors are acquired here using the \verb'baySeq::getLibsizes' function but might be acquired through any other means.

\begin{Schunk}
\begin{Sinput}
>   library(baySeq)
>   libsizes <- getLibsizes(data = ratThymus)
\end{Sinput}
\end{Schunk}

\section{K-means based clustering}

For k-means based clustering, we assume an approximately log-normal distribution. We adjust the data to remove zeros and rescale by the library scaling factors.
\begin{Schunk}
\begin{Sinput}
> ratThymus[ratThymus == 0] <- 1
> normRT <- log2(ratThymus %*% diag(1/libsizes) * mean(libsizes))
\end{Sinput}
\end{Schunk}

For speed purposes, we  will consider only the first 1000 genes in the data.
\begin{Schunk}
\begin{Sinput}
> normRT <- normRT[1:1000,]
\end{Sinput}
\end{Schunk}

K-means based clustering compares two genes by separately clustering the expression values of each gene using all possible values of $k$ (or, if the sample size is exceptionally large, all values of $k$ from 1 to some large maximum permitted value.). The genes are then compared to find the maximum $k$ for which the average expression of the identified clusters is monotonic between the two genes. For this value of $k$, the the maximum difference between expression levels observed within a cluster of either gene is reported as a measure of the dissimilarity between the two genes.

However, a problem arises for genes that are non-differentially expressed across all samples. For these genes, the appropriate number of clusters is 1; however, choices of $k > 1$ often lead to false assignment of genes to clusters exhibiting differential expression. To resolve this, the clustering incorporates a bootstrapping stage based on Tibshirani's gap statistic. Bootstrapping uniformly distributed data on the same range as the observed data, we calculate the the dissimilarity score as above, and find those cases for which the gap between the bootstrapped mean dissimilarity and the observed dissimilarity for k = 1 exceeds that for k = 2 by more than some multiple of the standard error of the bootstrapped dissimilarities of k = 2. These cases are forced to be treated as non-differentially expressed by discarding all dissimilarity data for k > 1.

We generate a matrix of dissimilarity scores between each pair of genes using the \verb'kCluster' function. The full matrix can be reported to (gzipped) file; the function returns a data.frame which for each gene defines its nearest neighbour of higher row index, and the dissimilarity with that neighbour. This is sufficient for a clustering based on singleton agglomeration (see \verb'hclust').

\begin{Schunk}
\begin{Sinput}
> kClust <- kCluster(normRT, matrixFile = "kclust_matrix.txt.gz", B = 1000)